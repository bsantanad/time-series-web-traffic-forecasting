\documentclass[journal]{IEEEtran}

% Packages
\usepackage{cite}
\usepackage{graphicx}
\usepackage{amsmath}
\usepackage{amssymb}
\usepackage{algorithm}
\usepackage{algorithmic}
\usepackage{hyperref}
\usepackage{svg}

% Title and author information
\title{Time Series for Web Traffic Forecasting }
\author{Benjamin Santana Velazquez}
\date{June 2023}

\begin{document}

\maketitle

\begin{abstract}
% Your abstract goes here
    Lorem ipsum dolor sit amet, consectetur adipiscing elit. Vestibulum
    dignissim felis in nisi lobortis, nec tincidunt nulla interdum. Nulla
    facilisi. Proin et augue risus. Nullam non ipsum in nisi accumsan bibendum.
    Curabitur sed elit at est aliquet convallis. Nullam dictum ullamcorper
    mauris id hendrerit. Mauris a dui tincidunt, feugiat magna vel, interdum
    sapien.
\end{abstract}

\begin{IEEEkeywords}
% Your keywords go here
    Forecasting
    Time Series
\end{IEEEkeywords}

\section{Introduction}
Every event serves as a salve to the progression of time, offering a compelling
rationale behind the significance of \emph{time series analysis} within the
realm of data science. In today's data-driven world companies hoard large
amounts of data over time, which contain imporant information about their
business patterns. Consequently, time series analysis has permeated various
fields, from the traditional ones like finance and economy --- where someone
can forecast sales or the EPS of a public company --- to lesser explored
territories like marketing, where time series analysis allows to track business
metrics, such as user engagement and make forecasts based on that. In this
piece of work we will delve into the realm of web traffic.

It is common for web applications to keep a register of their traffic at all
times. Meaning, if you are the owner of a website it is likely that you have
access to the data sent and received by your visitors. What is more, you
probably have access to the Clickstream Data of your users, which is the record
of an individual’s clicks through their journey on the Internet (in this case,
your website).

Taking advantage of this information, and the power of forecasting with
statistical models in time series, we can forecast the traffic of a website.

Why would we want to do this? \emph{Resource Allocation} and \emph{Performance
Optimization}. By forecasting website traffic, we can estimate the expected
number of visitors and allocate appropriate resources to handle the load. This
can either save costs in the infrastrucutre of the website like server
capacity, bandwith --- when a low number of visitors is present --- or it can
save the website's users from seeing an overload error when the server's
capacity is at peak.

Recently, a popular website by the name of \href{https://chess.com}{chess.com}
had this issue.  "Chess Is Booming! And Our Servers Are Struggling."
\cite{chesscom} was the title of an article posted after days of having their
servers at max capacity, not allowing users to play games and losing revenue in
the process.

This begs the question: Can we accurately forecast the traffic of the website
chess.com to enable proactive resource allocation and ensure sufficient
resources are allocated in advance? I will do my best to try and answer this in
this short paper

We will use one of the most popular traditional statistical methods used in
time series forecasting, $SARIMAX(p,d,q)(P,D,Q)_m$. We will find out if this
model is powerful enough to create meaningful and accurate forecasts or if
opting for naive methods to get better results.  we should opt into the world
of neural networks \footnote{We could also opt into the realm of neural network
for time series, but this is not covered in the paper, if you are interested
on that please refer to \cite{nn}}

Unfortunately, the taffic of a website is not public information. Meaning we
cannot get the actual time series from chess.com itself. We need to rely on
other sources to get the actual information. After reviewing the possible
options we decied to go with  \href{http://semrush.com}{semrush}, a "all-in-one
tool suite for improving online visibility and discovering marketing insights."
\cite{semrush}. One of semrush tools provides web traffic for different
websites. Of course we need to take into account that we are taking the data
(the time series itself) from a third party, so there is a some level of
uncertainty associated with its accuracy. Semrush provided us with the historic
information month by month of traffic in chess.com from January 2012 to March
2023. How it looks will be discussed later in the paper. What I wanted to point
here is that we will try to forecast 3 months into the future given these data
points

In summary, our objective is to assess the effectiveness of the
$SARIMAX(p,d,q)(P,D,Q)_m$ model in forecasting three months of web traffic on
chess.com. We will be conducting an analysis by comparing its performance
against naive forecasting methods. Such forecasts play a crucial role in
facilitating both resource allocation and performance optimization strategies
for the website. The conclution an results are shown in Section
\ref{sec:analysis} and Section \ref{sec:conclusion}


\section{Background Study}
% Discuss related work here
In this section we will define what a time series is and how it is composed.
We are also going to discuss the $MA(q)$ model and the $AR(p)$ model, which
serve as a base for $SARIMAX(p,d,q)(P,D,Q)_m$. Understanding these two parts
of the model --- I believe --- is the keystone to understanding the model as a
whole. We are also going to review the steps we are going to follow to tackle
the problem at hand. So let us start with a simple question...

What is a time series? We can define a time series as a set of points ordered
in time, usually equally spaced in time. \cite{timeseries} We can decompose
any time series into 3 components:

\begin{itemize}
    \item \emph{trend}, the slow-moving change in the time series.

    \item \emph{seasonal component}, a cycle that occurs over a fixed period
        of time.

    \item \emph{residuals}, what can not be explained by the trend of seasonal
        components, this is usually whtie noise. Any model wont be able to
        forecast any of this part, since it is completely random
\end{itemize}

Now that we have defined these concepts of a time series, let us see how they
llok in action. To demostrate how this looks in a time series we will use
classical time series presented in \emph{Box, G. E. P., Time Series Analysis,
Forecasting and Control.} \cite{airline}. The classic Box \& Jenkins airline
data presents monthly totals of international airline passengers, 1949 to
1960. Table~\ref{tab:passengerdata}, showcase the first 5 entries of the time
series.

\begin{table}[htbp]
  \centering
  \caption{First five rows of The classic Box \& Jenkins airline data}
  \label{tab:passengerdata}
  \begin{tabular}{|c|c|}
    \hline
    Month    & Passengers \\
    \hline
    1949-01  & 112        \\
    1949-02  & 118        \\
    1949-03  & 132        \\
    1949-04  & 129        \\
    1949-05  & 121        \\
    \hline
  \end{tabular}
\end{table}

If we plot the series, with the time in the $x$ axis and the passengers in
the $y$ axis, we get something like Figure~\ref{fig:airpassenger}. This is
what we call observed data.

\begin{figure}[htbp]
  \centering
  \includesvg[width=1\columnwidth]{images/airpassenger-ts.svg}
  \caption{The classic Box \& Jenkins airline data, it presents monthly
    totals of international airline passengers, 1949 to 1960.}
  \label{fig:airpassenger}
\end{figure}

Let us decompose it into the three components, to see how it would each would
look in a graph. Figure~\ref{fig:decompose} showcases how this looks.

\begin{figure}[htbp]
  \centering
  \includesvg[width=1\columnwidth]{images/decompose.svg}
  \caption{Decomposition of The classic Box \& Jenkins airline data, it
    presents monthly totals of international airline passengers, 1949 to
    1960.}
  \label{fig:decompose}
\end{figure}

We can see that now we have 4 plots displaying. The first one is the observed
data, just like we saw in Figure~\ref{fig:airpassenger}. Then we can see one
plot for each of the concepts explained above. We see the trend  as a
slow-change moving upwards. We can see the seasonal component repeating each
year, and finally the white noise or residuals of the time series.

Hopefully after seeing this example this concepts have clearer to the reader.

By now you might be wondering what are the differences between time series
forecasting and a other regression tasks. It basically comes down to two
things: \emph{time series has an order}, --- a time series is indexed by
time, order must be kept, in regression tasks when  you want to predict
revenue based on ad spend, it does not matter when certain amount was spent
on ads. --- and \emph{time series do not have features}, it is common to only
have two columns, time and the data itself, it usually does not have
categories.

Before jumping into $MA(q)$ and $AR(p)$ I want to explain some concepts, the
first one is \emph{baseline models}. A baseline model is a trivial solution
to our problem, it uses heuristics more than deep statsticts knowledge or
anything else. It might be clearer if we use an example. Say we have a time
series on the sales of XYZ, after getting the data points we calculate the
mean and say "I forecast next month our sales will be the mean of all the time
series". We got this forecast using a simple statistics concept like it is the
mean. It is likely that  we can get a better forecast than this using more
complex stastics. Why then should we bother making this naive models? They
allow us to compare our complex model with something. A common example of a
baseline model is just simply using the last value known. For example, if we
have five years worth of data points, collected each month, and we want to
forecast next three months, we can simply copy and paste the values from last
three months and be good to go. In some time series, our statistical model
maybe performs worse than this naive methods, then we would be wasting
resources every forecast running our complex statistical models. Because we
could just copy and past the last values, or use the mean, or some other
baseline model. In Section~\ref {sec:analysis} we will compare our
$SARIMAX(p,d,q)(P,D,Q)_m$ model with a baseline model and conclude if it is
worth it to use it against some simpler methods.

One last thing on the baseline methods. How do we compare the models against
each other? We will calculate an error metric in order to evaluate the
performance of our forecasts. We will use $MAPE$ (mean absolute percentage
error), which will measure prediction accuracy, independent of the scale of
our data

The next concept is \emph{stationarity}, a stationary time series is one whose
statistical properties do not change over time It has constant mean, variance,
and autocorrelation, and these properties are independent of time
\cite{timeseries}. Many models assume stationarity but we rarely see
stationarity in time series. Lucky for us we can transform the time series to
become stationary, there are many ways to transform it but the simplest (and
the one we are going to use) is \emph{differencing}. This consists on
calcluate the changes from one step to another.  This transformation helps
stabilize the mean. Which reduces the trend and seasonality effects. This
means that we also need to make sure we do an inverse transfrom of the data
after finishing the model.

We can test for stationarity with the \emph{Augmented Dickey-Fuller} (ADF) test
\cite{adf}, The key idea in the ADF test is to estimate an autoregressive model
of the time series and examine the significance of the coefficient on the
lagged first difference term. If the coefficient is significantly different
from zero, it suggests evidence against the presence of a unit root and
supports stationarity. \footnote{if you want more information on how this work
I highly recommend reading the paper where it was proposed \cite{adf}}
% add what the p value means

% explain MA and AR, and ARIMA

\section{Eperimental framework}
% Describe your methodology here

\section{Analysis of Results}\label{sec:analysis}
% Present your experimental results here

\section{Conclusion}\label{sec:conclusion}
% Summarize your findings and conclude the paper

\section*{Acknowledgment}
% Acknowledgments (if any) go here
I would like to express my sincere appreciation and gratitude to Marco
Peixeiro for his exceptional book, "Time Series Forecasting." This book has
been an invaluable resource throughout my research and study in the field of
time series analysis.

Marco Peixeiro's expertise and comprehensive coverage of time series
forecasting techniques have provided me with deep insights and practical
knowledge. The clear explanations and numerous examples presented in the book
have greatly enhanced my understanding of this complex subject.


\bibliographystyle{IEEEtran}
\bibliography{references}

\end{document}
