\documentclass[journal]{IEEEtran}
%\documentclass[10pt, twocolumn]{article}

% Packages
\usepackage{cite}
\usepackage{graphicx}
\usepackage{amsmath}
\usepackage{amssymb}
\usepackage{algorithm}
\usepackage{algorithmic}
\usepackage{hyperref}
\usepackage{svg}

% Title and author information
\title{Time Series for Web Traffic Forecasting }
\author{Benjamin Santana Velazquez}
\date{June 2023}

\begin{document}

\maketitle

\begin{abstract}
% Your abstract goes here
    Lorem ipsum dolor sit amet, consectetur adipiscing elit. Vestibulum
    dignissim felis in nisi lobortis, nec tincidunt nulla interdum. Nulla
    facilisi. Proin et augue risus. Nullam non ipsum in nisi accumsan bibendum.
    Curabitur sed elit at est aliquet convallis. Nullam dictum ullamcorper
    mauris id hendrerit. Mauris a dui tincidunt, feugiat magna vel, interdum
    sapien.
\end{abstract}

\begin{IEEEkeywords}
% Your keywords go here
    Forecasting
    Time Series
\end{IEEEkeywords}

\section{Introduction}\label{sec:intro}
Very few events are unaffected by time, this offers a compelling rationale
behind the significance of \emph{time series analysis} within the
realm of data science. In today's data-driven world companies hoard large
amounts of data over time, which contain imporant information about their
business patterns. Consequently, time series analysis has permeated various
fields, from the traditional ones like finance and economy --- where someone
can forecast sales or the EPS of a public company --- to lesser explored
territories like marketing, where time series analysis allows to track business
metrics, such as user engagement and make forecasts based on that. In this
piece of work we will delve into the realm of web traffic.

It is common for web applications to keep a register of their traffic at all
times. Meaning, if you are the owner of a website it is likely that you have
access to the data sent and received by your visitors. What is more, you
probably have access to the Clickstream Data of your users, which is the record
of an individual’s clicks through their journey on the Internet (in this case,
your website).

Taking advantage of this information, and the power of forecasting with
statistical models in time series, we can forecast the traffic of a website.

Why would we want to do this? \emph{Resource Allocation} and \emph{Performance
Optimization}. By forecasting website traffic, we can estimate the expected
number of visitors and allocate appropriate resources to handle the load. This
can either save costs in the infrastrucutre of the website like server
capacity, bandwith --- when a low number of visitors is present --- or it can
save the website's users from seeing an overload error when the server's
capacity is at peak.

Recently, a popular website by the name of \href{https://chess.com}{chess.com}
had this issue.  "Chess Is Booming! And Our Servers Are Struggling."
\cite{chesscom} was the title of an article posted after days of having their
servers at max capacity, not allowing users to play games and losing revenue in
the process.

This begs the question: Can we accurately forecast the traffic of the website
chess.com to enable proactive resource allocation and ensure sufficient
resources are allocated in advance? I will do my best to try and answer this in
this short paper

We will use one of the most popular traditional statistical methods used in
time series forecasting, $SARIMA(p,d,q)(P,D,Q)_m$. We will find out if this
model is powerful enough to create meaningful and accurate forecasts or if
opting for naive methods to get better results.  we should opt into the world
of neural networks \footnote{We could also opt into the realm of neural network
for time series, but this is not covered in the paper, if you are interested
on that please refer to \cite{nn}}

Unfortunately, the taffic of a website is not public information. Meaning we
cannot get the actual time series from chess.com itself. We need to rely on
other sources to get the actual information. After reviewing the possible
options we decied to go with  \href{http://semrush.com}{semrush}, a "all-in-one
tool suite for improving online visibility and discovering marketing insights."
\cite{semrush}. One of semrush tools provides web traffic for different
websites. Of course we need to take into account that we are taking the data
(the time series itself) from a third party, so there is a some level of
uncertainty associated with its accuracy. Semrush provided us with the historic
information month by month of traffic in chess.com from January 2012 to March
2023. How it looks will be discussed later in the paper. What I wanted to point
here is that we will try to forecast 3 months into the future given these data
points

In summary, our objective is to assess the effectiveness of the
$SARIMA(p,d,q)(P,D,Q)_m$ model in forecasting three months of web traffic on
chess.com. We will be conducting an analysis by comparing its performance
against naive forecasting methods. Such forecasts play a crucial role in
facilitating both resource allocation and performance optimization strategies
for the website. The conclution an results are shown in Section
\ref{sec:analysis} and Section \ref{sec:conclusion}


\section{Background Study}\label{sec:back}
% Discuss related work here
In this section we will define what a time series is and how it is composed.
We are also going to discuss the $MA(q)$ model and the $AR(p)$ model, which
serve as a base for $SARIMA(p,d,q)(P,D,Q)_m$. Understanding these two parts
of the model --- I believe --- is the keystone to understanding the model as a
whole. We are also going to review how $AR(p)$ and $MA(q)$ come together in
$ARMA(p,q)$ to model complex time series, and how we can find the orders $p$
and $q$. Let us start with a simple question...

What is a time series? We can define a time series as a set of points ordered
in time, usually equally spaced in time. \cite{timeseries} We can decompose
any time series into 3 components:

\begin{itemize}
    \item \emph{trend}, the slow-moving change in the time series.

    \item \emph{seasonal component}, a cycle that occurs over a fixed period
        of time.

    \item \emph{residuals}, what can not be explained by the trend of seasonal
        components, this is usually whtie noise. Any model wont be able to
        forecast any of this part, since it is completely random
\end{itemize}

Now that we have defined these concepts of a time series, let us see how they
llok in action. To demostrate how this looks in a time series we will use
classical time series presented in \emph{Box, G. E. P., Time Series Analysis,
Forecasting and Control.} \cite{airline}. The classic Box \& Jenkins airline
data presents monthly totals of international airline passengers, 1949 to
1960. Table~\ref{tab:passengerdata}, showcase the first 5 entries of the time
series.

\begin{table}[htbp]
  \centering
  \caption{First five rows of The classic Box \& Jenkins airline data}
  \label{tab:passengerdata}
  \begin{tabular}{|c|c|}
    \hline
    Month    & Passengers \\
    \hline
    1949-01  & 112        \\
    1949-02  & 118        \\
    1949-03  & 132        \\
    1949-04  & 129        \\
    1949-05  & 121        \\
    \hline
  \end{tabular}
\end{table}

If we plot the series, with the time in the $x$ axis and the passengers in
the $y$ axis, we get something like Figure~\ref{fig:airpassenger}. This is
what we call observed data.

\begin{figure}[htbp]
  \centering
  \includesvg[width=1\columnwidth]{images/airpassenger-ts.svg}
  \caption{The classic Box \& Jenkins airline data, it presents monthly
    totals of international airline passengers, 1949 to 1960.}
  \label{fig:airpassenger}
\end{figure}

Let us decompose it into the three components, to see how it would each would
look in a graph. Figure~\ref{fig:decompose} showcases how this looks.

\begin{figure}[htbp]
  \centering
  \includesvg[width=1\columnwidth]{images/decompose.svg}
  \caption{Decomposition of The classic Box \& Jenkins airline data, it
    presents monthly totals of international airline passengers, 1949 to
    1960.}
  \label{fig:decompose}
\end{figure}

We can see that now we have 4 plots displaying. The first one is the observed
data, just like we saw in Figure~\ref{fig:airpassenger}. Then we can see one
plot for each of the concepts explained above. We see the trend  as a
slow-change moving upwards. We can see the seasonal component repeating each
year, and finally the white noise or residuals of the time series.

Hopefully after seeing this example this concepts have clearer to the reader.

By now you might be wondering what are the differences between time series
forecasting and a other regression tasks. It basically comes down to two
things: \emph{time series has an order}, --- a time series is indexed by
time, order must be kept, in regression tasks when  you want to predict
revenue based on ad spend, it does not matter when certain amount was spent
on ads. --- and \emph{time series do not have features}, it is common to only
have two columns, time and the data itself, it usually does not have
categories.

Before jumping into $MA(q)$ and $AR(p)$ I want to explain some concepts, the
first one is \emph{baseline models}. A baseline model is a trivial solution
to our problem, it uses heuristics more than deep statsticts knowledge or
anything else. It might be clearer if we use an example. Say we have a time
series on the sales of XYZ, after getting the data points we calculate the
mean and say "I forecast next month our sales will be the mean of all the time
series". We got this forecast using a simple statistics concept like it is the
mean. It is likely that  we can get a better forecast than this using more
complex stastics. Why then should we bother making this naive models? They
allow us to compare our complex model with something. A common example of a
baseline model is just simply using the last value known. For example, if we
have five years worth of data points, collected each month, and we want to
forecast next three months, we can simply copy and paste the values from last
three months and be good to go. In some time series, our statistical model
maybe performs worse than this naive methods, then we would be wasting
resources every forecast running our complex statistical models. Because we
could just copy and past the last values, or use the mean, or some other
baseline model. In Section~\ref {sec:analysis} we will compare our
$SARIMA(p,d,q)(P,D,Q)_m$ model with a baseline model and conclude if it is
worth it to use it against some simpler methods.

One last thing on the baseline methods. How do we compare the models against
each other? We will calculate an error metric in order to evaluate the
performance of our forecasts. We will use $MAPE$ (mean absolute percentage
error), which will measure prediction accuracy, independent of the scale of
our data

The next concept is \emph{stationarity}, a stationary time series is one whose
statistical properties do not change over time It has constant mean, variance,
and autocorrelation, and these properties are independent of time
\cite{timeseries}. Many models assume stationarity but we rarely see
stationarity in time series. Lucky for us we can transform the time series to
become stationary, there are many ways to transform it but the simplest (and
the one we are going to use) is \emph{differencing}. This consists on
calcluate the changes from one time step to another ($y'_t = y_t - y_{t-1}$). This transformation helps
stabilize the mean. Which reduces the trend and seasonality effects. This
means that we also need to make sure we do an inverse transfrom of the data
after finishing the model.

We can test for stationarity with the \emph{Augmented Dickey-Fuller} (ADF) test
\cite{adf}, The key idea in the ADF test is to estimate an autoregressive model
of the time series and examine the significance of the coefficient on the
lagged first difference term. If the coefficient is significantly different
from zero, it suggests evidence against the presence of a unit root and
supports stationarity. \footnote{if you want more information on how this work
I highly recommend reading the paper where it was proposed \cite{adf}}

Basically the null hypothesis states that a unit root is present, meaning that
our time series is not stationary. We can reject the null hypothesis if after
doing the test, we get a $p$ value less than $0.05$.

I think we have a good enough grasp on time series to go into the moving
average process ($MA(q)$) and the autoregressive process ($AR(p)$)

\subsection{Moving Average Process}\label{sub:ma}

In a moving average ($MA$) process, the current value depens linearly on the
mean of the series, the current error term, and past error
terms.\cite{timeseries}. It is usually denoted as $MA(q)$ where $q$ is the
order, meaning the number of past error terms that affect the present value).

The moving average of order $q$ is calculated by equation~\ref{eq:ma}.
\begin{equation}
    y_t = \mu + \epsilon_t + \theta_1\epsilon_{t-1} + \theta_2\epsilon_{t-2} +
            ... +  \theta_q\epsilon_{t-q}
    \label{eq:ma}
\end{equation}

Where, $\mu$ is the mean, $\epsilon_t$ is the error term, $\epsilon_{t-k}$ is
the past error term and $\theta$ is the magnitude of impact of the past errors.

Needless to say, the large $q$ is, the more past error terms affect the
present value. This means that we need to choose this hyperparemeter carefully
in order to train the data correctly. In simple time series we can use simply
use the AutoCorrelation Function (ACF) plot in  order to get $q$. An example
of such plot is shown in Figure~\ref{fig:acf}.

\begin{figure}[htbp]
  \centering
  \includesvg[width=1\columnwidth]{images/acf.svg}
    \caption{Example ACF plot, where we see a relevant correlation up to lag 2}
  \label{fig:acf}
\end{figure}

In this plot we can see a relevant correlation up to lag 2, meaning we can use
$MA(2)$ to model this stationary time series process.

\subsection{Autoregressive Process}\label{sub:ar}

An autoregressive process ($AR(p)$)establishes that the output variable
depends linearly on its own previous states. An autoregressive process is a
regression of a variable against itself. In a time series, this means that the
present value is linearly dependent on its past values. \cite{timeseries}

It is defined by $AR(p)$ where $p$ is the order it defines the number of past
values that affect the present value. Equation~\ref{eq:ar} show an
autoregressive process up to lag $p$.

\begin{equation}
    y_t = C + \phi_1 y_{t-1} + \phi_2 y_{t-2}
            ... +  \phi_p y_{t-p} + \epsilon_t
    \label{eq:ar}
\end{equation}

Where $\phi$ is the autoregressive coefficients that measure the strength of
the relationship between the current value and past values, $\epsilon_t$ is
the error term and $\epsilon_{t-k}$ the past error term.

Like in subsection~\ref{sub:ma}, we also have an important hyperparemeter
here: $p$. Unfortunately we can not get it from the ACF plot, because if we
did an the plot on autoregressive stationary process it would exhibit a
pattern of exponential decay\cite{timeseries} like see in
Figure~\ref{fig:acf-ar}.

\begin{figure}[htbp]
  \centering
  \includesvg[width=1\columnwidth]{images/acf-ar.svg}
    \caption{Example ACF plot on an autoregressive process}
  \label{fig:acf-ar}
\end{figure}

We then must turn our attention to the Partial AutoCorrelation Function (PACF)
plot. to get the order of the autoregression. This measures the correlation
between lagged values in a time series when we remove the influence of
correlated lagged values in between. Figure~\ref{fig:pacf} showcase how this
would look.

\begin{figure}[htbp]
  \centering
  \includesvg[width=1\columnwidth]{images/pacf-ar.svg}
    \caption{Example PACF plot, We can see that we have no significant
        coeficients after lag 3, meaning $AR(p)$ where $p=3$, we have an
        autoregressive process of order 3}
    \label{fig:pacf}
\end{figure}

We can see in this example plot that we have no significant coeficients after
lag 3, therefore we can say that this process is $AR(3)$, meaning we have an
autoregressive process of order 3.

\subsection{Complex Time Series}

We have seen $MA(q)$ and $AR(p)$, but the curious reader (or one with a
background on time series analysis), will see that something is missing.  What
happens when the ACF plot shows a slowly decaying pattern or a sinusoidal
pattern, and, when plotting PACF you see too, a slowly decaying pattern or a
sinusoidal pattern. Meaning we cannot infer an order from the ACF plot or from
the PACF plot. Then, we may be in the precnese of a complex process, where we
will need to combine $AR(p)$ and $MA(q)$ to model it, this is where the
$ARMA(p, q)$ model comes into play.

The $ARMA(p, q)$  is simply a combination of the models we saw in
subsection~\ref{sub:ma} and subsection~\ref{sub:ar}. If we see how we can
model it, you will see that we basically have the two equations plugged
together. See equation~\ref{eq:arma} where:

\begin{equation}
\begin{aligned}
    y_t &= C + \phi_1 y_{t-1} + \phi_2 y_{t-2} +
            ... +  \phi_p y_{t-p} + \epsilon_t \\
        &+ \theta_1\epsilon_{t-1} + \theta_2\epsilon_{t-2} +
            ... +  \theta_q\epsilon_{t-q}
    \label{eq:arma}
\end{aligned}
\end{equation}

\begin{itemize}
    \item $p$ determines the number of past \emph{values} that affect the present
        value.
    \item $q$ determines the number of past \emph{errors} that affect the
        present value.
\end{itemize}

It is worth mentioning that $ARMA(0,q)$ process is equivalent to an $MA(q)$
process, since the order $p = 0$ cancels the $AR(p)$ portion. An $ARMA(p,0)$
process is equivalent to an $AR(p)$ process, since the order $q = 0$ cancels
the $MA(q)$ portion.

This still does not solve the issue I stated above, even if we combien both
models, we wont be able to get the order ($p$ and $q$) using the ACF and PACF
plot. Luckily for us we can use the \emph{Akaike Information Criterion}
($AIC$).

\subsection{Akaike Information Criterion}\label{sub:aic}

The $AIC$ estimates the quality of a model relative to other models.  Given
that there will be some information lost when a model is fitted to the data,
the $AIC$ quantifies the information lost. The less information lost, the
lower the $AIC$ value and the better the model.

Equation~\ref{eq:aic} shows how to calculate $AIC$.

\begin{equation}
    AIC = 2k - ln(\hat{L})
    \label{eq:aic}
\end{equation}

Where, $k$ is the number of estimated parameters, $\hat{L}$ maximum value of
the likehood function model.

Using $AIC$ to select our model, allows us to keep a balance between the
complexity of the model and its goodness of fitting the data. This is becase
$k$ is directly affected by the order of $p$ and $q$ in $ARMA(p,q)$. Let us
see an example; say $p = 2$ and $q=2$ then $k = p + q = 4$, since we are
substracting the likelihood function model to times two this number, the
higher the order gets, the $AIC$ increases, penalizng more complex models.

$\hat{L}$ (the likelihood function) measures the goodness of fit a model has.
Comparing it with the distribution function will make it easier to understand.
It can be seen as the opposite of the distribution function. In the
distribution function, given a model with fixed parameters, it measuers the
probability of obvserving a data point. In the likelihood function, given a
set of data, it tells us  how likely it is that different model parametrs
generate the data.

Let us see an example from the book Time Series Forecasting by Marco Peixeiroi
\cite{timeseries}, that helped me understand the concept better:

\begin{itemize}
    \item
        The distribution function tells us that there is a $\frac{1}{6}$
        probability that we'll observe ${1,2,3,4,5,6}$
    \item
        Suppose that we roll the die 10 times an you obtain
        $[1,5,3,4,6,2,4,3,2,1]$ The likelihood function will determine how
        likely is that the die has six sides.
\end{itemize}

So the question in our context is, \emph{How likely is that my observed data
is coming from an $ARMA(1,1)$} If a model fits the data really well, the
maximum value of likelihood will be high \cite{timeseries}. Since $AIC$
function subtracts the $ln(\hat{L})$ from $2k$, it balance between
underfitting and overfitting.

Taking into account that $AIC$ quantifies the quality of a model in relation
to other models only.  so it is a relative measure of quality.  What we end
upj doing in pracitce is iterating through some combinations of $p$ and $q$,
to calculate the $AIC$ values and selecting the one with the lowest score.
Findingin this way the order of $ARMA(p,q)$. We can extend this process to
even more complex models like $SARIMA(p,d,q)(P,D,Q)_m$, $AIC$ will help us
find this hyperparameters (except from $d$ and $D$, but more on that later).

Coming back to our inital problem (finding the order of $ARMA(p,q)$), after
finding a combination of $p$ and $q$, we still need to check one more thing.

\subsubsection{Calculating Residuals}\label{sub:resid}

The residuals of a model are simply the difference between the predicted
values and the actual values \cite{timeseries}.  If the process found by the
$ARMA(p,q)$ model is the same as the original process then the residuals would
be just $\epsilon_t$ (the error terms), meaning white noise, we cannot
preddict these values based on past values since they are complete random. In
summary, we want our residuals to behave like white noise (completely random).

We will need to check for two things, \emph{Q-Q plot} and the \emph{Ljung-Box
test}.

The quantile-quantile \emph{Q-Q} plot is a graphical technique for determining
if two data sets come from populations with a common distribution.  This is a
plot of the quantiles of the first data set against the quantiles of the
second data set. By a quantile, we mean the fraction (or percent) of points
below the given value. \cite{qq}

Basically we are going to plot the distribution of our residuals in one axis,
and a normal distribution in the other axis. If $y=x$ this means that our
residuls behave exactly like a normal distribution, meaning they are white
noise/completely random, so the order we chose (for $p$ and $q$) is good.

Luckily for us, the python module \emph{statsmodels} \cite{statsmodels},
allows us to do diagnostic plots for our residuals in a really easy way.

\begin{figure}[htbp]
  \centering
  \includesvg[width=1\columnwidth]{images/diag.svg}
    \caption{Example Diagnositc Plots for the residuals of a $ARMA(1,1)$ model}
  \label{fig:diag-example}
\end{figure}

Please refer to Figure~\ref{fig:diag-example}. In the top left we find a plot
of the residuals, we can see here that they are stationary, and they do not
have a trend. This is good, because that is the behaviout of a normal
distribution. In the top right, we can see a histogram of the residuals, it
looks like a normal distribution. In the bottom right, we can see the
autocorrelation function of the residuals, as we can see there is no
significant lag after lag 0, therefore there is no apparent autocorrelation.

Lastly on the bottom left, we found the Q-Q plot, as stated above, it is
comparing the distribution of our residuals with a normal distribtution. We
can see that in almost all instances $y=x$ meaning our residuals behave like a
normal distribution, therefore our residuals most be white noise (completely
random) which is what we expected.

After having a good Q-Q plot we go to Ljung-Box test to demonstrate that the
residuals are uncorrelated.

The Ljung-Box test is a statistical test that determines whether the
autocorrelation of a group of data is significantly different from 0.
The null hypothesis states that the data is independently distributed,
meaning that there is no autocorrelation. If the p-value is larger than
0.05, we cannot reject the null hypothesis meaning that the residuals are
independently distributed. \cite{timeseries}

Basically if in the Q-Q plot $y = x$ and when doing the Ljung-Box  test we get
on every lag $p > 0.05$ we can start using the model for forecasting.

\subsection{Brief Summary}

In this section we saw basic concepts of time series,  like decompostion and
stationarity. We learned about the $MA(q)$ and $AR(p)$ models, and how they
work together to form $ARMA(p,q)$ and model complex time series. We say how to
use $AIC$ to find the order of $p$ and $q$. We also saw how to analyise the
residuals to diagnose the model and see if we can use it in forecasting. It
goes without saying that this of course does not cover every single concept
related to $SARIMA(p,d,q)(P,D,Q)_m$. The idea of this section is to give an
understanding of the core concepts behind $SARIMA(p,d,q)(P,D,Q)_m$, to help
the reader follow the process we take in Section~\ref{sec:exp}. I encourage
the reader to refer to the refences we used throught this section for a
complete understanding of $SARIMA(p,d,q)(P,D,Q)_m$.

\section{Experimental framework}\label{sec:exp}
% Objective or Research Question
% Experimental Design
% Data Collection
% Variables
% Procedure

In this section we will FIXME

As stated in Section~\ref{sec:intro}, our objective is to assess the effectiveness of
the $SARIMA(p,d,q)(P,D,Q)_m$ model in forecasting three months of web traffic
on chess.com. We will be conducting an analysis by comparing its performance
against naive forecasting methods. To be more specific we will compare its
performance, against two \emph{naive methods} (also known as baseline models),
\emph{Last Known Value} and \emph{Window Average}.

Since we are trying to forecast 3 months into the future, for our first naive
method (\emph{Last Known Value}), we will take the values of the last 3 months
and use them as a forecast. For our second baseline model (\emph{Window
Average}) we will take a window of time, calculate the average and use that as
a forecast. In this case, that window of time will be the the last 6 months.
We will compare the models using the $MAPE$ function discussed in
Section~\ref{sec:back}.

In Section~\ref{sec:intro} we stated the traffic of a website is not public
information. We got the information from a thrid-party
\href{http://semrush.com}{semrush}. Semrush tools provides web traffic for
different websites. We need to take into account that we are taking the data
from a third party, so there is a some level of uncertainty associated with
its accuracy. Semrush provided us with the historic information month by month
of traffic in chess.com from January 2012 to March 2023. Since we are planning
to forecast three months into the future, we will split our data, in sets, one
for training and one for testing. We will take the information from Janurary
2012 up to December 2022 and make that our training set. While the first three
months of 2023 will serve as testing data. Meaning we will compare our
forecast with the actual information of these 3 months.

Let us start by outlining in general terms the process we will follow.

\begin{enumerate}
    \item Gather Data Points
    \item Decompose Time series
    \item Check for Stationarity
    \item Fit every combination of $SARIMA(p,d,q)(P,D,Q)_m$
    \item Select model with lowest $AIC$ \footnote{Refer to
        section~\ref{sub:aic} if you do not know what $AIC$ is.}
    \item Diagnose residuals
    \item Make forecast
\end{enumerate}

\subsection{Gather Data Points}

We downloaded the time series as a csv file, meaning it was easy to manipulate
using python. The first entries of the time series, look something like
Table~\ref{tab:traffic}

\begin{table}[htbp]
  \centering
  \caption{Traffic Data}
  \label{tab:traffic}
  \begin{tabular}{|c|c|c|}
    \hline
    Index & Date & Traffic \\
    \hline
    0 & 2012-01 & 526370 \\
    1 & 2012-02 & 489447 \\
    2 & 2012-03 & 482083 \\
    3 & 2012-04 & 497484 \\
    4 & 2012-05 & 531260 \\
    \hline
  \end{tabular}
\end{table}

We can see that the entries are collected monthly and on further inspection we
have no missing values. If we plot the date against the traffic we get
something like Figure~\ref{fig:obs}

\begin{figure}[htbp]
  \centering
  \includesvg[width=1\columnwidth]{images/obs.svg}
    \caption{Observed chess.com traffic from January 2012 to March 2023}
  \label{fig:obs}
\end{figure}

We can see a big spike on 2017, and an almost constat trend upward. For this
reason we will drop the values from before December 2016, since taking these
items into account will make our model forecast worse. We are left with 76
data points, which should be enough for our forecast.

As mentioned above, we will split the dataset, into training an testing
dataset. We need to take the last 3 months off from our dataset to get the
actual training data points. In Figure~\ref{fig:obs-2} we can see the dataset
once we have done this two changes.

\begin{figure}[htbp]
  \centering
  \includesvg[width=1\columnwidth]{images/obs-2.svg}
    \caption{Observed chess.com traffic from December 2016 to March 2023, with
        testing points grayed out}
    \label{fig:obs-2}
\end{figure}

\subsection{Decompose Time series}

Now that we have our time series ready let us analyse the decomposition of the
time series. \footnote{If you do not know what decomposing the time series
means please refer to Section~\ref{sec:back} or Figure~\ref{fig:decompose}}.
Please refer to Figure~\ref{fig:decom-chess}

\begin{figure}[htbp]
  \centering
  \includesvg[width=1\columnwidth]{images/decom.svg}
    \caption{Decomposed chess.com traffic from December 2016 to March 2023}
    \label{fig:decom-chess}
\end{figure}

In the first plot, top to bottom, we can see the observed data, which is the
same as in Figure~\ref{fig:obs-2}. Next plot we can see that the data has
mostly a slow moving change upwards, meaning it has a positive trend so most
likely we are not dealing with a stationary process. Moving one plot down we
will see the seasonality of the data, it does not seem to have any, there are
no apparent cycle that occurs over a fixed period of time.  If this is hard to
see, compare Figure~\ref{fig:decompose} with Figure~\ref{fig:decom-chess}. In
Figure~\ref{fig:decompose} we can see a clear cycle that occurs every year, in
Figure ~\ref{fig:decom-chess} we see nothing similar. This means that it is
likely our process does not has seasonality. Meaning all the part of
Seasonality in $SARIMA(p,d,q)(P,D,Q)_m$ can be discarded, we would end up with
$ARIMA(p,d,q)$. This reduces the combinations of hyperparameters we need to
calculate.\footnote{Please refer to subsection~\ref{sub:aic} to learn more
about how we find the hyperparameters ($ARIMA(p,d,q)$ order)}.

\subsection{Check for Stationarity}

We now know that it is likely that our time series is not stationary,
nevertheless let us run our \emph{Augmented Dickey-Fuller} (ADF) test.
\footnote{To learn more about the \emph{ADF} test please go back to
Section~\ref{sec:back} or to \cite{adf}}. The ADF statistic gave us $-0.34789$
and $p = 0.91844$ this means $p > 0.05$, therefore we cannot reject the null
hypothesis, and the series is not stationary. We use differencing to make it
stationary. After differencing it once we get a new ADF statistic of
$-3.18885$ and $p = 0.02063$ this means $p < 0.05$, therefore we can reject
the null hypothesis and say that the series is stationary after calculate the
first discrete difference. This means we have our first hyperparamet for
$ARIMA(p,d,q)$, $d = 1$ \footnote{$d$ is directly connected to the number of
times we have to apply a differenciation to make the series stationary, if
after applying the first differenciation $p > 0.05$ we would have to
differentiate again and $d$ would become 2}.

\subsection{Fit every combination of $SARIMA(p,d,q)(P,D,Q)_m$}

We have stated above that our model is $SARIMA(p,d,q)(P,D,Q)_m$ but with the
seasonality part set to zeros. Therefore we only need to find the combination
of the $p$ and $q$ orders. This becuase $P=0$, $D=0$, $Q=0$ since the time
series has no seasonality. We will try all the combinations generated by the
Cartesian product of $A\times A$ where $A$ is a set defined as $A =
\{0,1,2,3,4,5,6,7,8,9\}$. This means we will try $ARIMA(1,1,1)$, then
$ARIMA(1,1,2)$ up to $ARIMA(1,1,9)$, to then go $ARIMA(2,1,1)$ and try all
combinations. We will test each model with the $AIC$ go to
Section~\ref{sub:aic} for more information on how it works. Using this, we get
table, of which the first 5 entries are shown in Table~\ref{tab:aic}.

\begin{table}[htbp]
  \centering
  \caption{AIC Values}
  \label{tab:aic}
  \begin{tabular}{|c|c|c|}
    \hline
    Index & (p,q) & AIC \\
    \hline
    0 & (3, 1) & 2078.407462 \\
    1 & (0, 5) & 2079.774906 \\
    2 & (4, 1) & 2080.024761 \\
    3 & (3, 2) & 2080.353652 \\
    4 & (2, 3) & 2080.360632 \\
    5 & (0, 6) & 2081.280308 \\
    6 & (1, 5) & 2081.292037 \\
    \hline
  \end{tabular}
\end{table}

\subsection{Select Model with Lowest $AIC$}

We can see that our top choice is for $p=3$ and $q=1$. The second option, is
quite intresting it says $p=0$ and $q=5$, meaning a $MA(5)$ process might
model the time series if we make it stationary. This becuase if $p=0$ we can
take all the $AR(p)$ part of the process. It is worth exploring the top two
combinations and compare then against each other, see which one performs
better.

\subsection{Diagnose Residuals}

\subsubsection{$ARIMA(3,1,1)$}\label{sub:31}

\begin{figure}[htbp]
  \centering
  \includesvg[width=1\columnwidth]{images/diag-31.svg}
    \caption{Diagnostics plots for $ARIMA(3,1,1)$ residuals}
    \label{fig:diag-31}
\end{figure}

Please refer to Figure~\ref{fig:diag-31} while reading this subsection. In the
top left plot we can see that when plotted, the residuals seem to be
stationary, this is a good sign, becuase we do not want our residuals to have
any trend. On the top right plot we can see that the residuals behave similar
to a normal distribution. In the bottom right plot (Q-Q plot) we see that in
almost every point $y$ is similar to $x$, meaning the distribution of our
residuals and a normal distribution are similar. Finally in the bottom right
plot, we see that the residuals do no seem to have any correlation. So
everything points that our residuals behave like white noise, which is the
result we want.

When doing the Ljung-Box test, discussed in Section~\ref{sub:resid}, we saw
that at any given lag $p > 0.05$ meaning we cannot reject the null hypothesis,
therefore we can say that no lag is correlated.

\subsubsection{$ARIMA(0,1,5)$}

Please refer to Figure~\ref{fig:diag-05} while reading this subsection.
\begin{figure}[htbp]
  \centering
  \includesvg[width=1\columnwidth]{images/diag-05.svg}
    \caption{Diagnostics plots for $ARIMA(0,1,5)$ residuals}
    \label{fig:diag-05}
\end{figure}

The analysis for Figure~\ref{fig:diag-31} is pretty much the same as the one
presented in Subsection~\ref{sub:31} the only difference is on the Q-Q plot ,
and it is not significant so I wont spend time discussing it.

Similar to $ARIMA(3,1,1)$ when doing the Ljung-Box test, discussed in
Section~\ref{sub:resid}, we saw that at any given lag $p > 0.05$ meaning we
cannot reject the null hypothesis, therefore we can say that no lag is
correlated.

\subsection{Forecast}

\section{Analysis of Results}\label{sec:analysis}
% Present your experimental results here

\section{Conclusion}\label{sec:conclusion}
% Summarize your findings and conclude the paper

\section*{Acknowledgment}
% Acknowledgments (if any) go here
I would like to express my sincere appreciation and gratitude to Marco
Peixeiro for his exceptional book, "Time Series Forecasting." This book has
been an invaluable resource throughout my research and study in the field of
time series analysis.

Marco Peixeiro's expertise and comprehensive coverage of time series
forecasting techniques have provided me with deep insights and practical
knowledge. The clear explanations and numerous examples presented in the book
have greatly enhanced my understanding of this complex subject.

\bibliographystyle{IEEEtran}
\bibliography{references}

\end{document}
